\paragraph{\normalfont \large $[1]$  Kaminow, I.P., Li, T., et al. Optical fiber telecommunications. 5th Edition. Academic Press, Elsevier, San Diego (2008). \\ 
\\$[2]$ A. Naweed, G. Farca, S. I. Shopova, and A. T. Rosenberger "Induced transparency and absorption in coupled whispering-gallery microresonators", Phys. Rev. A \textbf{71} (2005)\\
\\$[3]$ B. Peng1, S. K. Ozdemir, W. Chen, F. Nori, L. Yang "What is and what is not electromagnetically induced transparency in whispering-gallery microcavities", Nature. Comm. (2014). \\
\\$[4]$ John E. Heebner, Ph.D. Thesis, "Nonlinear Optical Whispering Gallery Microresonators for Photonics", (2003)  \\
\\$[5]$ K. J. Vahala, “Optical microcavities,” Nature \textbf{424} (2003).\\
\\$[6]$ L. Maleki, A. B. Matsko, A. A. Savchenkov, and V. S. Ilchenko, “Tunable delay line with interacting
whispering-gallery-mode resonators,” Opt. Lett. 29(6), 626–628 (2004).\\
\\$[7]$ A. Naweed, D. Goldberg, and V. M. Menon, “All-optical electromagnetically induced transparency using
coupled one-dimensional microcavities,” Opt. Express 22, 18818–18823 (2014).\\
\\$[8]$ M. Borselli, T. Johnson, and O. Painter, “Beyond the Rayleigh scattering limit in high-Q silicon microdisks:
theory and experiment,” Opt. Express 13(5), 1515–1530 (2005).\\
\\$[9]$ Kobrinsky, M. J., Block, B.A., et al. On-chip optical interconnects. Intel Technol. J. \textbf{8}, 129 (2004).\\
\\$[10]$ Barwicz, T., Byun, H., et al. Silicon photonics for compact, energy-efficient interconnects. J. Opt. Networking \textbf{6}, 63 (2007)\\
\\$[11]$ Ishikawa, H. Ultrafast all-optical signal processing devices. John Wiley and Sons, New Jersey (2008). \\
\\$[12]$ Xia, F., Sekaric, L., et al. Ultracompact optical buffers on a silicon chip. Nature \textbf{1}, 65–71
(2007).\\
\\$[13]$ Landobasa, Y.M., Chin, M.K. Optical buffer with higher delay-bandwidth product in a tworing system. Opt. Express \textbf{16}, 1796–1807 (2008).\\
\\$[14]$ Fabry, C., Pérot, A. Théorie et applications d’une nouvelle méthode de spectroscopie interférentielle. Ann. Chim. Phys. \textbf{16}, 115 (1899).\\
\\$[15]$ M. Bayindir, S. Tanriseven, A. Aydinli, and E. Ozbay, “Strong enhancement of spontaneous emission in
amorphous-silicon-nitride photonic crystal based coupled-microcavity structures,” Appl. Phys., A Mater. Sci.
Process. \textbf{73}(1), 125–127 (2001)\\
\\$[16]$ M. Bayindir, S. Tanriseven, A. Aydinli, and E. Ozbay, “Strong enhancement of spontaneous emission in
amorphous-silicon-nitride photonic crystal based coupled-microcavity structures,” Appl. Phys., A Mater. Sci.
Process. \textbf{73}(1), 125–127 (2001).\\
\\$[17]$ A. J. Campillo, J. D. Eversole, and H.-B. Lin, “Cavity quantum electrodynamic enhancement of stimulated
emission in microdroplets,” Phys. Rev. Lett. \textbf{67}(4), 437–440 (1991).\\
\\$[18]$ D. Gerace, H. E. Türeci, A. Imamoglu, V. Giovannetti, and R. Fazio, “The quantum-optical Josephson
interferometer,” Nat. Phys. \textbf{5}(4), 281–284 (2009).\\
\\$[19]$  C. Diederichs, J. Tignon, G. Dasbach, C. Ciuti, A. Lemaître, J. Bloch, P. Roussignol, and C. Delalande,
“Parametric oscillation in vertical triple microcavities,” Nature \textbf{440}(7086), 904–907 (2006).\\
\\$[20]$ Q. Xu, S. Sandhu, M. L. Povinelli, J. Shakya, S. Fan, and M. Lipson, “Experimental realization of an on-chip alloptical analogue to electromagnetically induced transparency,” Phys. Rev. Lett. \textbf{96}(12), 123901 (2006).}

\paragraph{\normalfont \large $[21]$ J. Heebner, R. Grover, T. Ibrahim "Optical Microresonators, Theory, Fabrication, and Applications", Springer Science+Business Media (2008)\\
\\ $[22]$ K. Totsuka and M. Tomita "Dynamics of fast and slow pulse propagation through a microsphere–optical-fiber system", Phy. Rev. E \textbf{75} (2007)\\
\\ $[23]$ D. D. Smith, H. Chang, K. A. Fuller, A. T. Rosenberger, and R. W. Boyd, “Coupled-resonator-induced
transparency,” Phys. Rev. A \textbf{69}, 063804 (2004)\\
\\ $[24]$ A. J. Campillo, J. D. Eversole, and H.-B. Lin, “Cavity quantum electrodynamic enhancement of stimulated
emission in microdroplets,” Phys. Rev. Lett. \textbf{67}(4), 437–440 (1991).\\
\\ $[25]$ C. G. B. Garrett, and D. E. McCumber,  "Propagation of a Gaussian light pulse through an anomalous dispersion medium." Phys. Rev. A \textbf{1}, 305 (1970).\\
\\ $[26]$ Chu, S. and Wong, S. Linear pulse propagation in an absorbing medium. Phys. Rev. Lett. \textbf{48}, 738
(1982).\\
\\ $[27]$ Chiao, R. Y. Superluminal (but causal) propagation of wave packets in transparent media with inverted atomic populations. Phys. Rev. A \textbf{48}, R34 (1993).\\
\\ $[28]$ Bolda, E., Garrison, J. C. and Chiao, R. Y. Optical pulse propagation at negative group velocities due to a nearby gain line. Phys. Rev. A \textbf{49}, 2938 (1994).\\
\\ $[29]$ R.W. Boyd and D. Gauthier "Controlling the Velocity of Light Pulses", Science \textbf{326} (2009)\\
\\ $[30]$ L. J. Wang, A. Kuzmich and A. Dogariu "Gain-assisted superluminal light propagation", Nature \textbf{406} (2000)\\
\\ $[31]$  S. Y. Hu, E. R. Hegblom, and L. A. Coldren, “Coupled-cavity resonant-photodetectors for high-performance
wavelength demultiplexing applications,” Appl. Phys. Lett. \textbf{71}(2), 178–180 (1997).\\
\\ $[32]$  Z. Shi, R. W. Boyd, D. J. Gauthier, C. C. Dudley, “Enhancing the spectral sensitivity of interferometers
using slow-light media” Opt. Let. \textbf{32}, 8 (2007).\\
\\ $[33]$  M. Salit, G. S. Pati, K. Salit and M. S. Shahriar “Fast-light for astrophysics: super-sensitive gyroscopes and gravitational wave detectors” Journal of Modern Optics \textbf{54}, 16 (2007).\\
\\ $[34]$\,  Hecht, Jeff. The Laser Guidebook: Second Edition. McGraw-Hill, 1992. (Chapter 18-21).\\
\\ $[35]$  F. J. Duarte and L. W. Hillman (Eds.), Dye Laser Principles (Academic, New York, 1990).\\
\\ $[36]$ A. Naweed, "Photonic coherence effects from dual-waveguide coupled pair of co-resonant microring resonators", Opt. Exp. \textbf{23} (2015).\\
\\ $[37]$ Hau, L. V., Harris, S. E., Dutton, Z. and Behroozi, C. H. Light speed reduction to 17 meters per second in
an untracold atomic gas. Nature \textbf{397}, 594 (1999).\\
\\ $[38]$  Kash, M. M. et al. Ultraslow group velocity and enhanced nonlinear optical effects in a coherently
driven hot atomic gas. Phys. Rev. Lett. \textbf{82}, 5229 (1999).\\
\\ $[39]$ Budker, D., Kimball, D. F., Rochester, S. M. and Yashchuk, V. V. Nonlinear magneto-optics and reduced
group velocity of light in atomic vapor with slow ground state relaxation. Phys. Rev. Lett. \textbf{83}, 1767 (1999).\\
\\ $[40]$ Einstein, A., Lorentz, H. A., Minkowski, H. and Weyl, H. The Principle of Relativity, Collected Papers
(Dover, New York, 1952).}


\paragraph{\normalfont \large $[41]$ S. H. Autler and C. H. Townes, “Stark effect in rapidly varying fields,” Phys. Rev. \textbf{100} (1955) \\ 
\\$[42]$ S. E. Harris, "Electromagnetically Induced Transparency" Physics Today, July 1997 \\
\\$[43]$ X. Yang, M. Yu, D.-L. Kwong, and C. W. Wong, “All-optical analog to electromagnetically induced
transparency in multiple coupled photonic crystal cavities,” Phys. Rev. Lett. \textbf{102}(17), 173902 (2009). \\
\\$[44]$  O. Deparis and O. El Daif, “Optimization of slow light one-dimensional Bragg structures for photocurrent
enhancement in solar cells,” Opt. Lett. \textbf{37}(20), 4230–4232 (2012).\\
\\ $[45]$ B. Peng1, S. K. Ozdemir, W. Chen, F. Nori, L. Yang "What is and what is not electromagnetically induced transparency in whispering-gallery microcavities", Nature. Comm. (2014).\\
\\ $[46]$ Y.C. Liu, B.B. Li, and Y.F. Xiao "Electromagnetically induced transparency in optical microcavities", nanoph-2016-0168, (2017).\\
\\ $[47]$ S. Zhu, L. Shi, S. Yuan, R. Ma, X. Zhang and X. Fan, "All-optical controllable electromagnetically induced transparency in coupled silica microbottle cavities", nanoph-2018-0111 (2018).\\
\\ $[48]$ J. F. McMillan, X. Yang, N. C. Panoiu, R. M. Osgood, and C. W. Wong, “Enhanced stimulated Raman scattering
in slow-light photonic crystal waveguides,” Opt. Lett. \textbf{31}(9), 1235–1237 (2006).\\
\\ $[49]$ K. Totsuka, N. Kobayashi, and M. Tomita, “Slow light in coupled-resonator-induced transparency,” Phys. Rev.
Lett. \textbf{98}(21), 213904 (2007).\\
\\ $[50]$ A. Naweed (to be published)}

\paragraph{\normalfont \large $[51]$ S. H. Autler and C. H. Townes, “Stark effect in rapidly varying fields,” Phys. Rev. \textbf{100} (1955) \\ 
\\$[52]$  N. Miladinovic, F. Hasan, N. Chisholm, I. E. Linnington, E. A. Hinds, and D. H. J. O’Dell, “Adiabatic transfer of light in a double cavity and the optical Landau-Zener problem,” Phys. Rev. A \textbf{84}(4), 043822 (2011).\\
\\$[53]$ X. Yang, C. Husko, C. W. Wong, M. Yu, and D.-L. Kwong, “Observation of femtojoule optical bistability
involving Fano resonances in high-Q/V silicon photonic crystal nanocavities,” Appl. Phys. Lett. \textbf{91}(5), 051113
(2007).\\
\\$[54]$  V. M. Menon, W. Tong, and S. R. Forrest, “Control of quality factor and critical coupling in microring resonators
through integration of a semiconductor optical amplifier,” IEEE Photon. Technol. Lett. \textbf{16}(5), 1343–1345 (2004).\\
\\$[55]$ C. Diederichs, J. Tignon, G. Dasbach, C. Ciuti, A. Lemaître, J. Bloch, P. Roussignol, and C. Delalande,“Parametric oscillation in vertical triple microcavities,” Nature \textbf{440}(7086), 904–907 (2006).\\
\\$[56]$ D. O’Brien, A. Gomez-Iglesias, M. D. Settle, A. Michaeli, M. Salib, and T. F. Krauss, “Tunable optical delay using photonic crystal heterostructure nanocavities ,” Appl. Phys. Rev. B \textbf{76}, 115110 (2007).\\
\\$[57]$ C. Yang,  X. Jiang Q. Hua, S. Hua, Y. Chen, and M. Xiao, “Realization of controllable photonic molecule based on three ultrahigh-Q microtoroid cavities,” Laser Photonics Rev., 2017, \textbf{11},\\
\\$[58]$ X. Yang, M. Yu, D.-L. Kwong, and C. W. Wong, “All-optical analog to electromagnetically induced
transparency in multiple coupled photonic crystal cavities,” Phys. Rev. Lett. \textbf{102}(17), 173902 (2009).\\
\\$[59]$ D. Cui, C. Xie, Y. Liu, Y. Li, L. Wei, Y. Wang, J. Liu, and C. Xue, “Experimental demonstration of inducedtransparency based on a novel resonator system,” Opt. Commun. 324, 296–300 (2014).\\
\\$[60]$ A. Naweed (private communications).}