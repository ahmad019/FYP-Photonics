\section{EIT in Atoms}
Most of the time, using mpmath is simply a matter of setting the desired precision and entering a formula. For verification purposes, a quite (but not always!) reliable technique is to calculate the same thing a second time at a higher precision and verifying that the results agree.
\subsection{Two level Atoms}
To perform more advanced calculations, it is important to have some understanding of how mpmath works internally and what the possible sources of error are. This section gives an overview of arbitrary-precision binary floating-point arithmetic and some concepts from numerical analysis.Most of the time, using mpmath is simply a matter of setting the desired precision and entering a formula. For verification purposes, a quite (but not always!) reliable technique is to calculate the same thing a second time at a higher precision and verifying that the results agree.
\section{EIT in ring resonators}
To perform more advanced calculations, it is important to have some understanding of how mpmath works internally and what the possible sources of error are. This section gives an overview of arbitrary-precision binary floating-point arithmetic and some concepts from numerical analysis.Most of the time, using mpmath is simply a matter of setting the desired precision and entering a formula. For verification purposes, a quite (but not always!) reliable technique is to calculate the same thing a second time at a higher precision and verifying that the results agree.
\section{EIT in Coupled resonators(CRIT)}
To perform more advanced calculations, it is important to have some understanding of how mpmath works internally and what the possible sources of error are. This section gives an overview of arbitrary-precision binary floating-point arithmetic and some concepts from numerical analysis.Most of the time, using mpmath is simply a matter of setting the desired precision and entering a formula. For verification purposes, a quite (but not always!) reliable technique is to calculate the same thing a second time at a higher precision and verifying that the results agree.
\section{CRIT with gain}
To perform more advanced calculations, it is important to have some understanding of how mpmath works internally and what the possible sources of error are. This section gives an overview of arbitrary-precision binary floating-point arithmetic and some concepts from numerical analysis.Most of the time, using mpmath is simply a matter of setting the desired precision and entering a formula. For verification purposes, a quite (but not always!) reliable technique is to calculate the same thing a second time at a higher precision and verifying that the results agree.
\section{Results}
To perform more advanced calculations, it is important to have some understanding of how mpmath works internally and what the possible sources of error are. This section gives an overview of arbitrary-precision binary floating-point arithmetic and some concepts from numerical analysis.Most of the time, using mpmath is simply a matter of setting the desired precision and entering a formula. For verification purposes, a quite (but not always!) reliable technique is to calculate the same thing a second time at a higher precision and verifying that the results agree.

To perform more advanced calculations, it is important to have some understanding of how mpmath works internally and what the possible sources of error are. This section gives an overview of arbitrary-precision binary floating-point arithmetic and some concepts from numerical analysis.