\section{The Fabry-Perot Interferometer}
Most of the time, using mpmath is simply a matter of setting the desired precision and entering a formula. For verification purposes, a quite (but not always!) reliable technique is to calculate the same thing a second time at a higher precision and verifying that the results agree.
\subsection{Theory of Fabry-Perot interferometer}
To perform more advanced calculations, it is important to have some understanding of how mpmath works internally and what the possible sources of error are. This section gives an overview of arbitrary-precision binary floating-point arithmetic and some concepts from numerical analysis.Most of the time, using mpmath is simply a matter of setting the desired precision and entering a formula. For verification purposes, a quite (but not always!) reliable technique is to calculate the same thing a second time at a higher precision and verifying that the results agree.
\subsection{Finese, Q-factor}
To perform more advanced calculations, it is important to have some understanding of how mpmath works internally and what the possible sources of error are. This section gives an overview of arbitrary-precision binary floating-point arithmetic and some concepts from numerical analysis.Most of the time, using mpmath is simply a matter of setting the desired precision and entering a formula. For verification purposes, a quite (but not always!) reliable technique is to calculate the same thing a second time at a higher precision and verifying that the results agree.
\section{Gain incorporation in Resonators}
To perform more advanced calculations, it is important to have some understanding of how mpmath works internally and what the possible sources of error are. This section gives an overview of arbitrary-precision
\subsection{Beer's Law}
 binary floating-point arithmetic and some concepts from numerical analysis.Most of the time, using mpmath is simply a matter of setting the desired precision and entering a formula. For verification purposes, a quite (but not always!) reliable technique is to calculate the same thing a second time at a higher precision and verifying that the results agree.
\subsection{Beer's law study as gain}
To perform more advanced calculations, it is important to have some understanding of how mpmath works internally and what the possible sources of error are. This section gives an overview of arbitrary-precision binary floating-point arithmetic and some concepts from numerical analysis.Most of the time, using mpmath is simply a matter of setting the desired precision and entering a formula. For verification purposes, a quite (but not always!) reliable technique is to calculate the same thing a second time at a higher precision and verifying that the results agree.
\section{Gain medium}
To perform more advanced calculations, it is important to have some understanding of how mpmath works internally and what the possible sources of error are. This section gives an overview of arbitrary-precision binary floating-point arithmetic and some concepts from numerical analysis.