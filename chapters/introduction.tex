\section{Resonators}
Most of the time, using mpmath is simply a matter of setting the desired precision and entering a formula. For verification purposes, a quite (but not always!) reliable technique is to calculate the same thing a second time at a higher precision and verifying that the results agree.
\subsection{Explaination}
To perform more advanced calculations, it is important to have some understanding of how mpmath works internally and what the possible sources of error are. This section gives an overview of arbitrary-precision binary floating-point arithmetic and some concepts from numerical analysis.Most of the time, using mpmath is simply a matter of setting the desired precision and entering a formula. For verification purposes, a quite (but not always!) reliable technique is to calculate the same thing a second time at a higher precision and verifying that the results agree.
\section{Optical Resonators}

To perform more advanced calculations, it is important to have some understanding of how mpmath works internally and what the possible sources of error are. This section gives an overview of arbitrary-precision binary floating-point arithmetic and some concepts from numerical analysis.Most of the time, using mpmath 
\subsection{Different Geometeries}

is simply a matter of setting the desired precision and entering a formula. For verification purposes, a quite (but not always!) reliable technique is to calculate the same thing a second time at a higher precision and verifying that the results agree.
\section{Fabry-Perot Resonators}

To perform more advanced calculations, it is important to have some understanding of how mpmath works internally and what the possible sources of error are. This section gives an overview of arbitrary-precision binary floating-point arithmetic and some concepts from numerical analysis.
\subsection{Explaination}
To perform more advanced calculations, it is important to have some understanding of how mpmath works internally and what the possible sources of error are. This section gives an overview of arbitrary-precision binary floating-point arithmetic and some concepts from numerical analysis.

\section{Ring Resonators}
To perform more advanced calculations, it is important to have some understanding of how mpmath works internally and what the possible sources of error are. This section gives an overview of arbitrary-precision binary floating-point arithmetic and some concepts from numerical analysis.
\subsection{All-Pass}
To perform more advanced calculations, it is important to have some understanding of how mpmath works internally and what the possible sources of error are. This section gives an overview of arbitrary-precision binary floating-point arithmetic and some concepts from numerical analysis.
\subsection{Add drop}
To perform more advanced calculations, it is important to have some understanding of how mpmath works internally and what the possible sources of error are. This section gives an overview of arbitrary-precision binary floating-point arithmetic and some concepts from numerical analysis.
\subsection{Coupled Ring}
To perform more advanced calculations, it is important to have some understanding of how mpmath works internally and what the possible sources of error are. This section gives an overview of arbitrary-precision binary floating-point arithmetic and some concepts from numerical analysis.


\section*{References}
\addcontentsline{toc}{section}{References}

To perform more advanced calculations, it is important to have some understanding of how mpmath works internally and what the possible sources of error are. This section gives an overview of arbitrary-precision binary floating-point arithmetic and some concepts from numerical analysis.
To perform more advanced calculations, it is important to have some understanding of how mpmath works internally and what the possible sources of error are. This section gives an overview of arbitrary-precision binary floating-point arithmetic and some concepts from numerical analysis.
