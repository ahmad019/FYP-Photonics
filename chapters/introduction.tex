\section{Resonators}
\normalfont \large A resonator is a device that exhibits resonant behavior naturally (or artificially) on some resonant frequencies, that is, it oscillates at those frequencies with higher amplitudes than others. These frequencies are called its resonant frequencies. These oscillations can either be electromagnetic waves or mechanical waves as well. 
There are different uses of resonators, they can be used to filter some specific frequencies or can also be used to generate a specific frequency of the wave. A resonator in which the waves exists in hallow space is called a cavity resonator, which is used in electronics and radio signal processing,  known as microwave cavities, to generate, transmit and receive electromagnetic signals.  Acoustic cavity resonators, in which sound is produced by air vibrating in a cavity with one opening, are known as Helmholtz resonators.
\subsection{Explaination}
The term resonator is most often used for a homogeneous object in which vibrations travel as waves, at an approximately constant velocity, bouncing back and forth between the sides of the resonator. The material of the resonator, through which the waves flow, can be viewed as being made of millions of coupled moving parts (such as atoms). Therefore, they can have millions of resonant frequencies, although only a few may be used in practical resonators. The oppositely moving waves interfere with each other, and at its resonant frequencies reinforce each other to create a pattern of standing waves in the resonator. If the distance between the sides is ${\displaystyle d\,}$, the length of a round trip is ${\displaystyle 2d\,}$. To cause resonance, the phase of a sinusoidal wave after a round trip must be equal to the initial phase so the waves self-reinforce. The condition for resonance in a resonator is that the round trip distance, ${\displaystyle 2d\,}$, is equal to an integer number of wavelengths ${\displaystyle \lambda \,}$ of the wave:

$${\displaystyle 2d=N\lambda ,\qquad \qquad N\in \{1,2,3,\dots \}}$$

If the velocity of a wave is ${\displaystyle c\,}$, the frequency is ${\displaystyle f=c/\lambda \,}$ so the resonant frequencies are:

$${\displaystyle f={\frac {Nc}{2d}}\qquad \qquad N\in \{1,2,3,\dots \}}$$

So the resonant frequencies of resonators, called normal modes, are equally spaced multiples (harmonics) of a lowest frequency called the fundamental frequency. The above analysis assumes the medium inside the resonator is homogeneous, so the waves travel at a constant speed, and that the shape of the resonator is rectilinear. If the resonator is inhomogeneous or has a nonrectilinear shape, like a circular drumhead or a cylindrical microwave cavity, the resonant frequencies may not occur at equally spaced multiples of the fundamental frequency. They are then called overtones instead of harmonics. There may be several such series of resonant frequencies in a single resonator, corresponding to different modes of vibration.

\section{Optical Resonators}

To perform more advanced calculations, it is important to have some understanding of how mpmath works internally and what the possible sources of error are. This section gives an overview of arbitrary-precision binary floating-point arithmetic and some concepts from numerical analysis.Most of the time, using mpmath 
\subsection{Different Geometeries}

is simply a matter of setting the desired precision and entering a formula. For verification purposes, a quite (but not always!) reliable technique is to calculate the same thing a second time at a higher precision and verifying that the results agree.
\section{Fabry-Perot Resonators}

To perform more advanced calculations, it is important to have some understanding of how mpmath works internally and what the possible sources of error are. This section gives an overview of arbitrary-precision binary floating-point arithmetic and some concepts from numerical analysis.
\subsection{Explaination}
To perform more advanced calculations, it is important to have some understanding of how mpmath works internally and what the possible sources of error are. This section gives an overview of arbitrary-precision binary floating-point arithmetic and some concepts from numerical analysis.

\section{Ring Resonators}
To perform more advanced calculations, it is important to have some understanding of how mpmath works internally and what the possible sources of error are. This section gives an overview of arbitrary-precision binary floating-point arithmetic and some concepts from numerical analysis.
\subsection{All-Pass}
To perform more advanced calculations, it is important to have some understanding of how mpmath works internally and what the possible sources of error are. This section gives an overview of arbitrary-precision binary floating-point arithmetic and some concepts from numerical analysis.
\subsection{Add drop}
To perform more advanced calculations, it is important to have some understanding of how mpmath works internally and what the possible sources of error are. This section gives an overview of arbitrary-precision binary floating-point arithmetic and some concepts from numerical analysis.
\subsection{Coupled Ring}
To perform more advanced calculations, it is important to have some understanding of how mpmath works internally and what the possible sources of error are. This section gives an overview of arbitrary-precision binary floating-point arithmetic and some concepts from numerical analysis.


\section*{References}
\addcontentsline{toc}{section}{References}

To perform more advanced calculations, it is important to have some understanding of how mpmath works internally and what the possible sources of error are. This section gives an overview of arbitrary-precision binary floating-point arithmetic and some concepts from numerical analysis.
To perform more advanced calculations, it is important to have some understanding of how mpmath works internally and what the possible sources of error are. This section gives an overview of arbitrary-precision binary floating-point arithmetic and some concepts from numerical analysis.
